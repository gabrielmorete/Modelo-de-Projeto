%%%%%%%%%% Arquivo de configuração do projeto
%% Insira seus dados e modifique o modelo com suas preferências
%% Comente/descomente linhas para modificar o modelo
%% Aviso! O modelo não segue uma norma específica

%%%%%%%%%%%%%% Informações dos projeto
\tipo{Projeto de Pesquisa para Iniciação Científica}
\titulo{Nome do Projeto}
\aluno{Nome do Aluno}
\emailaluno{aluno@universidade.br}
\orientador{Nome do Orientador}
\emailorientador{orietador@universidade.br}
\palavraschave{Projeto, IC, Otimização Combinatória}
\data{\today}

%%%%%%%%%%%%%% Pacotes extras
\usepackage{lipsum} % gera texto dummy

%%%%%%%%%%%%%% Bibliografia
%% Para referências estilo ABNT tire o comentário de uma das linhas 
%% Manual do pacote de referências : https://www.abntex.net.br/

\usepackage[num]{abntex2cite}	% Referências numéricas padrão ABNT
%\usepackage[alf]{abntex2cite}  % Referências alfabéticas padrão ABNT

\AtEndDocument{
    \bibliografia{bibtex.bib} % Arquivo com o bibtex
    \estilobibliografico{plain} % Estilo das referências se não usar ABNT
    % www.openoffice.org/bibliographic/bibtex-defs.html
}

%% Para adicionar backrefs tire o comentário da linha abaixo,
%% backrefs listam as paginas em que uma referência foi citada
%\usepackage[brazilian,hyperpageref]{backref} % Adiciona backrefs

%%%%%%%%%%%%%% Confs
\AtBeginDocument{
    \maketitle % Gera cabeçalho
    \pagestyle{plain} % Efetivamente, insere a numeração
}

%%%%%%%%%%%%%%%% Comandos úteis para matemática
\newcommand{\Z}{\mathbb{Z}}
\newcommand{\Q}{\mathbb{Q}}
\newcommand{\R}{\mathbb{R}}
\newcommand{\trans}{^\intercal}

\DeclareMathOperator{\posto}{posto}
\DeclareMathOperator{\capacidade}{cap}
\DeclareMathOperator{\valor}{val}
\DeclareMathOperator{\mdc}{mdc}
\DeclareMathOperator{\adj}{adj}
\DeclareMathOperator{\conv}{conv}
\DeclareMathOperator{\cone}{cone}

\DeclarePairedDelimiter\ceil{\lceil}{\rceil}
\DeclarePairedDelimiter\floor{\lfloor}{\rfloor}

\newtheorem{teorema}{Teorema}
\newtheorem{definicao}{Definição}
\newtheorem{lema}{Lema}
\newtheorem{proposicao}{Proposição}
\newtheorem{corolario}{Corolário}
\newtheorem{fato}{Fato}
\theoremstyle{definition}