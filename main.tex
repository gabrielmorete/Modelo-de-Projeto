\documentclass[12pt]{projeto}

% Insira suas informações no arquivo conf.tex
%%%%%%%%%% Arquivo de configuração do projeto
%% Insira seus dados e modifique o modelo com suas preferências
%% Comente/descomente linhas para modificar o modelo
%% Aviso! O modelo não segue uma norma específica

%%%%%%%%%%%%%% Informações dos projeto
\tipo{Projeto de Pesquisa para Iniciação Científica}
\titulo{Nome do Projeto}
\aluno{Nome do Aluno}
\emailaluno{aluno@universidade.br}
\orientador{Nome do Orientador}
\emailorientador{orietador@universidade.br}
\palavraschave{Projeto, IC, Otimização Combinatória}
\data{\today}

%%%%%%%%%%%%%% Pacotes extras
\usepackage{lipsum} % gera texto dummy

%%%%%%%%%%%%%% Bibliografia
%% Para referências estilo ABNT tire o comentário de uma das linhas 
%% Manual do pacote de referências : https://www.abntex.net.br/

%\usepackage[num]{abntex2cite}	% Referências numéricas padrão ABNT
%\usepackage[alf]{abntex2cite}  % Referências alfabéticas padrão ABNT

\AtEndDocument{
    \bibliography{bibtex.bib} % Arquivo com o bibtex
    \bibliografia{plain} % Estilo das referências se não usar ABNT
    % www.openoffice.org/bibliographic/bibtex-defs.html
}

%% Para adicionar backrefs tire o comentário da linha abaixo,
%% backrefs listam as paginas em que uma referência foi citada
%\usepackage[brazilian,hyperpageref]{backref} % Adiciona backrefs

%%%%%%%%%%%%%% Confs
\AtBeginDocument{
    \maketitle % Gera cabeçalho
    \pagestyle{plain} % Efetivamente, insere a numeração
}

%%%%%%%%%%%%%%%% Comandos úteis para matemática
\newcommand{\Z}{\mathbb{Z}}
\newcommand{\Q}{\mathbb{Q}}
\newcommand{\R}{\mathbb{R}}
\newcommand{\trans}{^\intercal}

\DeclareMathOperator{\posto}{posto}
\DeclareMathOperator{\capacidade}{cap}
\DeclareMathOperator{\valor}{val}
\DeclareMathOperator{\mdc}{mdc}
\DeclareMathOperator{\adj}{adj}
\DeclareMathOperator{\conv}{conv}
\DeclareMathOperator{\cone}{cone}

\DeclarePairedDelimiter\ceil{\lceil}{\rceil}
\DeclarePairedDelimiter\floor{\lfloor}{\rfloor}

\newtheorem{teorema}{Teorema}
\newtheorem{definicao}{Definição}
\newtheorem{lema}{Lema}
\newtheorem{proposicao}{Proposição}
\newtheorem{corolario}{Corolário}
\newtheorem{fato}{Fato}
\theoremstyle{definition} % Arquivo de configuração

\begin{document}

\begin{resumo}
    \lipsum[1]
\end{resumo}

\section{Introdução}

\ifthenelse{\value{__usandoabnt} < 1}{
      blabla  
    }{bananas}
\lipsum[2]

\begin{align*}
    \sum_{e \in E(T)} c(e) + \sum_{v \notin V(T)} \pi(v)
\end{align*}

\lipsum[3 - 4]

Em 1972, Karp \cite{Karp72} mostrou que alguns problemas de decisão associados a problemas de otimização são $\mathcal{NP}$-completo e, com isso, o problema de otimização é $\mathcal{NP}$-difícil. Dessa forma, não é esperado que se encontre um algoritmo polinomial para tal problema. Assim, podemos citar alguns artigos \cite{BolukbasiK18, dimacs11, LjubicWPKMF05}, para exemplificar a bibliografia no final do documento.

\section{Objetivos e justificativa}

\lipsum[1]

Em síntese, pretende-se que, ao final do projeto, o aluno esteja familiarizado com os seguintes tópicos:

\begin{enumerate}[label=\roman*)]
    \item lorem ipsum dolor sit amet, consectetuer adipiscing eli;
    \item ut purus elit, vestibulumut, placerat ac, adipiscing vitae, feli;
    \item curabitur dictum gravida mauris;
    \item donec vehicula augue euneque.
\end{enumerate}

\section{Métodos}

\lipsum[2]

\subsection{Lorem ipsum dolor sit amet}

\lipsum[3]

\section{Plano de trabalho e cronograma}

\lipsum[4]

Abaixo segue uma tabela resumindo o que foi descrito acima. Cada célula representa um bimestre.

\begin{table}[ht] % Dica : https://www.tablesgenerator.com/latex_tables
\centering
\begin{tabular}{|c|c|c|c|c|c|c|}
    \hline
    \multirow{2}{*}{Atividade}            
    & \multicolumn{6}{c|}{Bimestre} \\ \cline{2-7} & 1 & 2 & 3 & 4 & 5 & \multicolumn{1}{l|}{6} \\ \hline
    Atividade 1 & $\times$ & $\times$ & $\times$ & & & \\ \hline
    Atividade 2 & & $\times$ & $\times$ & $\times$ & & \\ \hline
    Atividade 3 & & & & $\times$ & $\times$ & $\times$ \\ \hline
\end{tabular}
\caption{Cronograma das atividades.}
\end{table}

\section{O candidato}

\lipsum[5]

\end{document}
